\documentclass[11pt,preprint, authoryear]{elsarticle}

\usepackage{lmodern}
%%%% My spacing
\usepackage{setspace}
\setstretch{1.2}
\DeclareMathSizes{12}{14}{10}{10}

% Wrap around which gives all figures included the [H] command, or places it "here". This can be tedious to code in Rmarkdown.
\usepackage{float}
\let\origfigure\figure
\let\endorigfigure\endfigure
\renewenvironment{figure}[1][2] {
    \expandafter\origfigure\expandafter[H]
} {
    \endorigfigure
}

\let\origtable\table
\let\endorigtable\endtable
\renewenvironment{table}[1][2] {
    \expandafter\origtable\expandafter[H]
} {
    \endorigtable
}


\usepackage{ifxetex,ifluatex}
\usepackage{fixltx2e} % provides \textsubscript
\ifnum 0\ifxetex 1\fi\ifluatex 1\fi=0 % if pdftex
  \usepackage[T1]{fontenc}
  \usepackage[utf8]{inputenc}
\else % if luatex or xelatex
  \ifxetex
    \usepackage{mathspec}
    \usepackage{xltxtra,xunicode}
  \else
    \usepackage{fontspec}
  \fi
  \defaultfontfeatures{Mapping=tex-text,Scale=MatchLowercase}
  \newcommand{\euro}{€}
\fi

\usepackage{amssymb, amsmath, amsthm, amsfonts}

\def\bibsection{\section*{References}} %%% Make "References" appear before bibliography


\usepackage[round]{natbib}

\usepackage{longtable}
\usepackage[margin=2.3cm,bottom=2cm,top=2.5cm, includefoot]{geometry}
\usepackage{fancyhdr}
\usepackage[bottom, hang, flushmargin]{footmisc}
\usepackage{graphicx}
\numberwithin{equation}{section}
\numberwithin{figure}{section}
\numberwithin{table}{section}
\setlength{\parindent}{0cm}
\setlength{\parskip}{1.3ex plus 0.5ex minus 0.3ex}
\usepackage{textcomp}
\renewcommand{\headrulewidth}{0.2pt}
\renewcommand{\footrulewidth}{0.3pt}

\usepackage{array}
\newcolumntype{x}[1]{>{\centering\arraybackslash\hspace{0pt}}p{#1}}

%%%%  Remove the "preprint submitted to" part. Don't worry about this either, it just looks better without it:
\makeatletter
\def\ps@pprintTitle{%
  \let\@oddhead\@empty
  \let\@evenhead\@empty
  \let\@oddfoot\@empty
  \let\@evenfoot\@oddfoot
}
\makeatother

 \def\tightlist{} % This allows for subbullets!

\usepackage{hyperref}
\hypersetup{breaklinks=true,
            bookmarks=true,
            colorlinks=true,
            citecolor=blue,
            urlcolor=blue,
            linkcolor=blue,
            pdfborder={0 0 0}}


% The following packages allow huxtable to work:
\usepackage{siunitx}
\usepackage{multirow}
\usepackage{hhline}
\usepackage{calc}
\usepackage{tabularx}
\usepackage{booktabs}
\usepackage{caption}


\newenvironment{columns}[1][]{}{}

\newenvironment{column}[1]{\begin{minipage}{#1}\ignorespaces}{%
\end{minipage}
\ifhmode\unskip\fi
\aftergroup\useignorespacesandallpars}

\def\useignorespacesandallpars#1\ignorespaces\fi{%
#1\fi\ignorespacesandallpars}

\makeatletter
\def\ignorespacesandallpars{%
  \@ifnextchar\par
    {\expandafter\ignorespacesandallpars\@gobble}%
    {}%
}
\makeatother


\urlstyle{same}  % don't use monospace font for urls
\setlength{\parindent}{0pt}
\setlength{\parskip}{6pt plus 2pt minus 1pt}
\setlength{\emergencystretch}{3em}  % prevent overfull lines
\setcounter{secnumdepth}{5}

%%% Use protect on footnotes to avoid problems with footnotes in titles
\let\rmarkdownfootnote\footnote%
\def\footnote{\protect\rmarkdownfootnote}
\IfFileExists{upquote.sty}{\usepackage{upquote}}{}

%%% Include extra packages specified by user

%%% Hard setting column skips for reports - this ensures greater consistency and control over the length settings in the document.
%% page layout
%% paragraphs
\setlength{\baselineskip}{12pt plus 0pt minus 0pt}
\setlength{\parskip}{12pt plus 0pt minus 0pt}
\setlength{\parindent}{0pt plus 0pt minus 0pt}
%% floats
\setlength{\floatsep}{12pt plus 0 pt minus 0pt}
\setlength{\textfloatsep}{20pt plus 0pt minus 0pt}
\setlength{\intextsep}{14pt plus 0pt minus 0pt}
\setlength{\dbltextfloatsep}{20pt plus 0pt minus 0pt}
\setlength{\dblfloatsep}{14pt plus 0pt minus 0pt}
%% maths
\setlength{\abovedisplayskip}{12pt plus 0pt minus 0pt}
\setlength{\belowdisplayskip}{12pt plus 0pt minus 0pt}
%% lists
\setlength{\topsep}{10pt plus 0pt minus 0pt}
\setlength{\partopsep}{3pt plus 0pt minus 0pt}
\setlength{\itemsep}{5pt plus 0pt minus 0pt}
\setlength{\labelsep}{8mm plus 0mm minus 0mm}
\setlength{\parsep}{\the\parskip}
\setlength{\listparindent}{\the\parindent}
%% verbatim
\setlength{\fboxsep}{5pt plus 0pt minus 0pt}



\begin{document}



\begin{frontmatter}  %

\title{Impact of United States Dollar Volatility Across Emerging and Developed
Equity Markets}

% Set to FALSE if wanting to remove title (for submission)




\author[Add1]{Ntai Lesenya}
\ead{22356258@sun.ac.za}





\address[Add1]{22356258}
\address[Add2]{DEPARTMENT OF ECONOMICS}
\address[Add3]{STELLENBOSCH UNIVERSITY}


\begin{abstract}
\small{
The paper seeks to assess the presence and impact of a unidirectional
flow of volatility from the United States Dollar (USD) to five Developed
and five Emerging Markets equity returns across the Global Financial
Crisis (GFC) period. Basing its analysis on the Morgan Stanley Capital
International (MSCI) equity returns indexes, the paper utilizes
multivariate GARCH Dynamic Conditional Correlation (DCC) processes to
isolate the dynamic conditional correlations from the conditional
variance component in order to highlight and rank these markets'
sensitivity to the USD volatility. Generally, there is evidence of low
asymmetric dynamic correlation between USD volatility and equity returns
volatility from both markets. This implies that exchange rate and equity
markets are informationally inefficient such that one market has some
predictive power over the other. The bivariate correlation is greater
among developed markets and lowest among emerging markets. The analysis
indicates that South Africa in particular has the lowest bivariate
conditional correlation with the USD compared to all other equity
markets, thus proving to be a preferable diversification choice in times
of global market uncertainty. These results have substantial
implications for international portfolio managers and hedgers in their
assessment of investment opportunities and asset diversification
decisions across markets. Consequently, Investors can better comprehend
how volatility between exchange rate and equity markets interlink
overtime in order to forecast market behavior.
}
\end{abstract}

\vspace{1cm}

\begin{keyword}
\footnotesize{
Multivariate GARCH - Dynamic Conditional Correlation Model
\sep Volatility Spillover \sep Exchange Rate market \sep Equity Markets \\ \vspace{0.3cm}
\textit{JEL classification} L250 \sep L100
}
\end{keyword}
\vspace{0.5cm}
\end{frontmatter}



%________________________
% Header and Footers
%%%%%%%%%%%%%%%%%%%%%%%%%%%%%%%%%
\pagestyle{fancy}
\chead{}
\rhead{}
\lfoot{}
\rfoot{\footnotesize Page \thepage}
\lhead{}
%\rfoot{\footnotesize Page \thepage } % "e.g. Page 2"
\cfoot{}

%\setlength\headheight{30pt}
%%%%%%%%%%%%%%%%%%%%%%%%%%%%%%%%%
%________________________

\headsep 35pt % So that header does not go over title




\hypertarget{introduction}{%
\section{\texorpdfstring{Introduction
\label{Introduction}}{Introduction }}\label{introduction}}

Volatility modeling is one area that has received considerable attention
especially with regard to equity market and exchange rate volatility
correlations. The internationalization of capital markets and
homogenization of asset price movements across financial systems is one
factor that has fueled interest in exchange rate volatility and its
effect on stock market volatility. According to Katzke
(\protect\hyperlink{ref-katzke2013}{2013}), strong correlation of asset
markets that go even beyond fundamental linkages tend to be rife during
global economic uncertainty as a result of these interconnected within
global financial systems and instantaneous information flow . The
working theory has been that the increased globalization of financial
markets contributes to the increase in volatility of financial systems
since these systems have been shown to be susceptible to surprise shocks
from other markets including exchange rate fluctuations.

Investors have over the years taken note of the fact that they trade off
expected return and risk as a result of the inherent volatility in these
markets. Understanding this risk-return trade-off is fundamental to
investors as they endeavor to diversify their portfolios by investing in
different asset classes which have low to negative correlations across
equity markets. It is well documented that emerging equity markets are
characterized by high volatility as opposed to developed markets. The
study looks at the intensity at which the US dollar volatility
contributes to the EM and DM stock market returns volatility. The paper
can complement the emerging body of literature by establishing which
among developed and emerging equity markets is mostly vulnerable to the
US dollar volatility.

Moreover, by relating US dollar volatility effect in emerging markets,
to its volatility effect in developed markets, the paper can assess
whether the effects in different markets are driven by these common
component or other macroeconomic or political factors. The results can
enhance the systematic understanding of spillover activities between the
US as a super power and the rest of the world. To capture the volatility
spillover the study employs DCC GARCH modeling technic because of its
capability to capture time-varying correlations in volatility. It finds
that currency movements can have significant effects on investment
returns volatility which has led to the use of currency hedges to remove
direct effects of exchange rate volatility in some instances. Hence it
concludes that evaluation of potential returns from either developed or
emerging stock markets should take into account an analysis of currency
trends on company earnings as this bare vital insight into global asset
performance.

The remainder of the paper is organized as follows. Section 2 highlights
relevant literature to both exchange rate and equity market volatility
among EM and DM countries. Section 3 breakdown the data composition for
the data used in the study and the associated quality control measures
employed to ensure accuracy and appropriateness of the intended
analysis. Section 4 details the methodology adopted for analysis of the
subject at hand while Section 5 presents the results of the estimated
models. The last Section presents a summary and conclusion of the study.

\hypertarget{theoretical-overview}{%
\section{\texorpdfstring{Theoretical Overview
\label{Theorectical Overview}}{Theoretical Overview }}\label{theoretical-overview}}

In theory, investment decisions are governed by Portfolio theory which
dictates that an investor should firstly be guided by the desire to
maximize expected return while minimizing risk. These strategies are
however at loggerheads since riskier assets usually bare higher returns
whereas less risky assets bare low returns. The direction of correlation
between any two assets also has bearing on their riskiness. Negative
correlation between risky assets reduces portfolio volatility since a
negative return is accompanied by a positive return hence a reduction in
the portfolio's volatility whereas a positive correlation between risky
assets increases portfolio volatility (Ruppert \& Matteson,
\protect\hyperlink{ref-ruppert2015}{2015}). Since investors demand a
reward for bearing the risk they usually opt for volatile assets in
expectation of the associated higher returns, thus investors are drawn
towards the more volatile markets. Emerging markets have for the longest
time been theorized to be the more volatile market and a common
destination for international portfolio investors looking to obtain
diversification rewards compared to developed markets (Hung,
\protect\hyperlink{ref-hung2019}{2019}). The latter has been proven by
the surge of investors to EMs in recent years as increased number of
high return shares were issued, reflected by the growing weight of EMs
in the MSCI all country index composition (Blitz \emph{et al.},
\protect\hyperlink{ref-blitz2013}{2013}). The dispersion of volatility
among EM countries also tends to be higher than that found among DM
countries whereas cross correlation among EMs is small.

According to Bekaert \& Harvey
(\protect\hyperlink{ref-bekaert1997}{1997}) this only forms part of the
distinguishing factors between developed and emerging markets. He
explains that emerging countries also have high average returns, low
market return correlations with developed markets and more predictable
returns (Bekaert \& Harvey, \protect\hyperlink{ref-bekaert1997}{1997}).
DeSantis \& Imrohoroglu (\protect\hyperlink{ref-desantis1994}{1994}) is
also off the view that EM rather than DMs are characterized by high
kurtosis as a result of returns that varied from either very high or
very low. Shin (\protect\hyperlink{ref-shin2005}{2005}) also emphasize
that the degree of integration between EMs and DMs plays an important
role in determining spillover effects between them since a shock to the
world market returns affects all countries with any form of covariance
with it. The source of volatility maybe related to asset diversification
or concentration, equity market development, macroeconomic or political
influences (Bekaert \& Harvey,
\protect\hyperlink{ref-bekaert1997}{1997}).But for purposes of this
paper, the focus is on exchange rate volatility, as a source of
volatility with particular interest on US dollar volatility because of
its position as the global reserve currency and dominance of the US
financial market. An appreciation of the domestic currency against the
dollar has an influence on international investors because their returns
increase whereas a depreciation of the local currency against the dollar
results in a reduction of their returns. According to Bahmani-Oskooee \&
Sohrabian (\protect\hyperlink{ref-bahmani1992}{1992}), the inverse
relationship between domestic (foreign) assets and foreign (domestic)
interest rates and the positive relationship between domestic (foreign)
assets and domestic (foreign) interest rates solidify the role of
exchange rates in balancing asset demand and supply.

Lim \& Sek (\protect\hyperlink{ref-lim2014}{2014}) studies Volatility in
EMS and concludes that exchange rate volatility does affect equity
market returns. Adjasi \emph{et al.}
(\protect\hyperlink{ref-adjasi2008}{2008}) also shares the same view and
finds an inverse relationship between exchange rate volatility and
equity market returns. They note that an increase in volatility of the
local currency exchange value results in financial volatility and
subsequent increase in investor uncertainty. Mcgibany \& Nourzad
(\protect\hyperlink{ref-mcgibany1995}{1995}) reiterate that when faced
with such an option, local investors are motivated to substitute assets
they believe are safe for riskier. They however note that Investors tend
to seek hedging against their own currency depreciation, or even seek to
profit off other currencies' appreciation by holding assets linked to it
in order to cushion against volatility risk. Kanas
(\protect\hyperlink{ref-kanas2000}{2000}) \& Morales
(\protect\hyperlink{ref-morales2008}{2008}) study on volatility
spillover from exchange rate markets to equity markets in USA, UK,
Canada, Japan, Germany, France, Czech Republic, Hungary, Poland and
Slovakia using bivariate Exponential-GARCH (EGARCH) model on daily data
and find insignificant volatility spillover from exchange rate to equity
markets. This they explained could be a result of a number of reasons.
Firstly the use of daily data which couldn't capture effects of trade
flows on exchange rate changes, secondly, the counteractive effect of
positive exchange rate volatility on equity markets for one firm can be
offset by a negative effect for others resulting in weak exchange rate
effect and lastly the reason could be the fact that volatility
spillovers can be counteracted by the use of exchange rate risk hedges
such as futures, forwards and currency option (Kanas,
\protect\hyperlink{ref-kanas2000}{2000}). Raghavan \& Dark
(\protect\hyperlink{ref-raghavan2008}{2008}) however find a significant
unidirectional volatility spillover from the USD/AUD exchange rates to
the Australian All Ordinaries Index (AOI) using daily data from 2
January 1995 to 31 December 2004. However, Mishra \emph{et al.}
(\protect\hyperlink{ref-mishra2007}{2007}) \& Kumar
(\protect\hyperlink{ref-kumar2013}{2013}) use the bivariate EGARCH model
in India, Brazil, and South Africa and find a significant bi-directional
volatility spillover between stock and foreign exchange markets proving
the existence of information transmission and integration between the
two markets. Moreover,Chkili \& Nguyen
(\protect\hyperlink{ref-chkili2014}{2014}) uses Markov-Switching EGARCH
model on weekly data to find a significant impact of exchange rate
changes on stock market volatility in Hong Kong, Singapore, Malaysia and
Mexico. They also conclude that the relationship between exchange rate
and equity markets in EMs dependent on either low volatility or high
volatility regimes.

Findings from the various studies differ from country to country
conditional on the methodology and time span of data utilized. It is
therefore prudent to study the effects of exchange rate volatility on
equity market returns with particular interest on comparing samples from
EMs and DMs within the same context, using updated methodologies, time
span and most importantly using recent data. Therefore the importance of
modeling volatility within financial markets has taken center stage
since the introduction of ARCH models by Engle
(\protect\hyperlink{ref-engle1982}{1982}) in his seminal paper (Katzke,
\protect\hyperlink{ref-katzke2013}{2013}; Silvennoinen \& Teräsvirta,
\protect\hyperlink{ref-silvennoinen2009}{2009}). Silvennoinen \&
Teräsvirta (\protect\hyperlink{ref-silvennoinen2009}{2009}) shows that
the initial works of authors like Bollerslev \emph{et al.}
(\protect\hyperlink{ref-bollerslev1994}{1994}) \& Shephard
(\protect\hyperlink{ref-shephard1996}{1996}) focused on univariate
models hence extending such works to the multivariate dimension is also
vital. Ruppert \& Matteson (\protect\hyperlink{ref-ruppert2015}{2015})
shows that Generalized Autoregressive Conditional Heteroskedasticity
(GARCH) on the other hand have gained popularity in econometrics because
of their ability to randomly varying volatility in time series data.
These models are best known to model both the conditional
heteroskedasticity and the heavy-tailed distributions of equity markets
data.

\hypertarget{data}{%
\section{\texorpdfstring{Data \label{Data}}{Data }}\label{data}}

The study uses value-weighted equity market indices for ten major equity
markets around the world. These are Egypt, Taiwan, India, Brazil, South
Africa, Japan, United Kingdom, France, Canada and Switzerland. The data
is acquired from Morgan Stanley Capital International (MSCI) for the
period starting in 2003 to 2018 (15-year period to capture the impact of
the Global Financial Crisis (GFC)). This sample period enables the
research to examine the effect of US dollar volatility on EM-DM equity
markets during both bad and good times. The GFC is an appropriate
example of bad times since both Developed countries and Emerging
countries' equity markets experienced a downturn.

Under the MSCI taxonomy, Egypt, Taiwan, India, Brazil, South Africa, are
categorised as `emerging markets while the other countries are
categorised as developed' markets. Bloomberg on the other hand employes
an Index that tracks the performance of a basket of 10 leading global
currencies against the United states dollar. This is expressed as the
Bloomberg Dollar Spot Index(BBDXY). All these indices are widely used in
equity market research on volatility transmission because of the degree
of comparability they allow. Weekly data is analysed as opposed to
monthly data in order to provide sufficient observation required by the
Garch model in the absence of short term correlations due to the noise
that is usually found in daily data. Higher frequency data like daily
data tends to be more volatile and has more noise that might not
correspond to long run fundamental behaviour. Monthly data on the other
hand does not give sufficient information for estimation since monthly
returns are longer horizon returns which can musk momentary responses to
innovations which may only last for days. The stock value indexes are
expressed in US dollars to avoid exchange rates fluctuations. For
further analysis, the continuously compounded daily returns are
calculated by taking the log difference of each listed company as
follows;

\[
r_{i,t} = ln (\frac{P_{i,t}}{P_{i,t-1}})*100  
\]

Where \(r_{i,t}\) represents the natural logarithm of weekly stock
returns from selected DM and EM country indices with \(P_{i,t}\) as the
closing price of the market index, i, at time t.

Table \ref{tab1} outlines all the EM and DM countries and their
respective Tickers as used by the MSCI database. Table \ref{tab2} on the
other hand presents the summary Statistics for all the tickers used for
analysis. The normality test using the Jarque--Bera test the weekly
returns is rejected as expected. In addition, both the skewness and
excess kurtosis statistics also indicate that the distributions of all
the Indices are non-normal. That is the kurtosis is greater than zero
which represents a normal distribution. The excess kurtosis shows that
the distribution has heavy tails and as such is a leptokurtic
distribution common in financial time series data.The reurns as seen in
table \ref{tab2} are substancially negatively skewed. The BBDXY indice
displays the highest mean whereas the MXJP displays the lowest mean. The
graphical representation of all the indices can be viewed on figure
\ref{figure1} in the Appendix . All the equty markets were substantially
affect by the 2008 Global Financial Crisis as seen be the decline in
returns during the period. The recurrence of very high and very low
returns observed particularly in the sampled emerging countries
indicates the presence of leptokurtosis. The overall graphicall
presentation in figure \ref{figure1} indicate that periods of volatility
clustering highlighling the presence of serial heteroskedasticity.

In order to control for any remaining serial heteroskedasticity in the
series and explore the conditional covariance structure for further
analysis the next section shows the generalization of the univariate
Garch(1,1) Model to the multivariate domain in order to conduct
multivariate volatility modeling. \hfill

\begin{table}[H]
\centering
\begin{tabular}{rlrrrrrrrrr}
  \hline
 & group1 & mean & sd & median & trimmed & min & max & skew & kurtosis & se \\ 
  \hline
dlogret1 & BBDXY & 0.00 & 0.01 & 0.00 & 0.00 & -0.04 & 0.04 & 0.26 & 1.77 & 0.00 \\ 
  dlogret2 & MXBR & 0.00 & 0.05 & 0.01 & 0.00 & -0.33 & 0.26 & -0.45 & 5.92 & 0.00 \\ 
  dlogret3 & MXCA & 0.00 & 0.03 & 0.00 & 0.00 & -0.26 & 0.18 & -1.17 & 10.69 & 0.00 \\ 
  dlogret4 & MXCH & 0.00 & 0.03 & 0.00 & 0.00 & -0.25 & 0.13 & -1.53 & 14.67 & 0.00 \\ 
  dlogret5 & MXEG & 0.00 & 0.04 & 0.00 & 0.00 & -0.37 & 0.15 & -1.45 & 9.59 & 0.00 \\ 
  dlogret6 & MXFR & 0.00 & 0.03 & 0.00 & 0.00 & -0.27 & 0.14 & -1.12 & 8.10 & 0.00 \\ 
  dlogret7 & MXGB & 0.00 & 0.03 & 0.00 & 0.00 & -0.28 & 0.17 & -1.35 & 13.84 & 0.00 \\ 
  dlogret8 & MXIN & 0.00 & 0.04 & 0.00 & 0.00 & -0.22 & 0.18 & -0.45 & 3.82 & 0.00 \\ 
  dlogret9 & MXJP & 0.00 & 0.02 & 0.00 & 0.00 & -0.16 & 0.09 & -0.64 & 3.61 & 0.00 \\ 
  dlogret10 & MXZA & 0.00 & 0.04 & 0.00 & 0.00 & -0.20 & 0.29 & 0.09 & 4.43 & 0.00 \\ 
  dlogret11 & TAMSCI & 0.00 & 0.03 & 0.00 & 0.00 & -0.12 & 0.10 & -0.52 & 1.53 & 0.00 \\ 
   \hline
\end{tabular}
\caption{Descriptive Statistics Table \label{tab2}} 
\end{table}

\hypertarget{methodology}{%
\section{\texorpdfstring{Methodology
\label{Meth}}{Methodology }}\label{methodology}}

In order to analyze the transmission of volatility from the US dollar to
both EM and DM equity markets, direct time-varying conditional
correlations between the different markets are modeled using Dynamic
Conditional Correlation (DCC) MV-GARCH models as suggested by Engle
(\protect\hyperlink{ref-engle2002}{2002}). These models are preferred
due to their ability to accommodate volatility clustering. Engle
(\protect\hyperlink{ref-engle2002}{2002}) \& Katzke
(\protect\hyperlink{ref-katzke2020}{2020}) emphasize that these models
are preferable due to their applicability in various situations where
their flexibility enables specifying univariate GARCH Models and leads
to gains in parsimony as they can be estimated either with univariate or
two step methods based on the likelihood function to keep track of the
time evolution of conditional correlations regardless of the number of
assets(Tsay, \protect\hyperlink{ref-tsay2013}{2013}). The first step is
to obtain the GARCH estimates for the univariate volatility estimates
for each returns series. The empirical analysis is undertaken
considering a stochastic process \(r_{it}\) of a continuously compounded
weekly returns as shown below;

\begin{align}
r_{it} = \mu_{i,t} + \varepsilon_{i,t}  \label{eq1} \\ \notag 
\end{align}

Where \(\mu_{i,t}\) is the conditional mean and \(\epsilon_{i,t}\) is
the conditionally heteroskedastic error series.\(\epsilon_{it}\) is
expressed as \(\epsilon_{it}= \sqrt h_{it} \eta_{it}\) and can be used
to fit the univariate GARCH process for each series to obtain the
conditional variance used to standardise the residuals. This is
expressed as;

\begin{align}
\eta_{it}= \epsilon_{it}/ \sqrt h_{it} \quad with \quad \epsilon_{it} \sim N(0,H_t)\quad and\quad \eta_{it} \sim N(0,I)   \label{eq2} \\ \notag 
\end{align}

On a multivariate scale, according to (Engle,
\protect\hyperlink{ref-engle2002}{2002}) we generalise Bollerslev (1990)
Constant Conditional Correlation (CCC) estimator and use the
standardised residuals from \ref{eq2} above to construct time varying
conditional correlations whereby firstly the dynamic correlation model
\(H_t\) is assumed to be;

\begin{align} 
H_t = D_t.R_t.D_t. \label{eq3} \\ \notag
\end{align}

Equation \ref{eq3} splits the variance Covariance matrix into identical
diagonal matrices \(D_t\) and refers to time varying conditional
correlations matrix \(R_t\). The diagonal matrices are defined as;

\begin{align} 
D_t = diag(\sqrt h_{it}). \label {eq4} \\ \notag
 \end{align}

Then we define the dynamic conditional correlation structure. This is
shown in \ref{eq5} below;

\begin{align}  \label {eq5}
Q_{ij,t} &= \bar Q + \alpha\left(z_{t - 1}z'_{t - 1} - \bar{Q} \right) + \beta\left( Q_{ij, t - 1} - \bar{Q} \right) \hfill \\ \notag  &= (1 - \alpha - \beta)\bar{Q} + \alpha z_{t - 1}z'_{t - 1} + \beta Q_{ij, t - 1} \notag
\end{align}

\(Q_{ij,t}\) is the unconditional variance between series i and j.
\(\bar{Q}\) is the unconditional covarience between univariate series
estimated in step 1. The non-negative parameters are represented by
\(\alpha\) and \(\beta\) and they must satisfy \(\alpha + \beta < 1\)
and each is \textless{} 0 so that the model is mean reverting Engle
(\protect\hyperlink{ref-engle2002}{2002}). The paper however takes note
of the fact that this could possibly represent a drawback of the DCC
models since the \(\alpha\) and \(\beta\) being scalars could mean that
all the conditional correlations will obey the same dynamics which are
necessary to ensure that \(R_t\) is positive definite as Bauwens
\emph{et al.} (\protect\hyperlink{ref-bauwens2006}{2006}) suggests.
Mishra \emph{et al.} (\protect\hyperlink{ref-mishra2007}{2007}) also
suggests that GARCH models do not differentiate reaction of volatility
from either positive and negative shocks. He however argues that
negative shock to financial time series could cause a rise in volatility
that is greater than a positive shock of the same magnitude.

Employing Equation \ref{eq5} the dynamic time-varying conditional
correlation matrix \(R_t\) is estimated as;

\begin{align}   \label{eq6}
R_t &= diag(Q_t)^{-1/2}Q_t.diag(Q_t)^{-1/2}. 
\end{align}

Equation \ref{eq6} allows us to fit an \(R_t\) matrix with the following
elements;

\begin{align} \label{eq7}
R_t &= \rho_{ij,t} = \frac{q_{i,j,t}}{\sqrt{q_{ii,t}.q_{jj,t}}} 
\end{align}

This results in a DCC model formulated as in \ref{eq8}, which has the
assumption of normality that produces a likehood estimator as opposed to
a Quasi-maximum Likelood estimator and also includes the assumption that
each asset in the series follows a univariate GARCH process(Engle,
\protect\hyperlink{ref-engle2002}{2002}).

\begin{align} \label{eq8}
\varepsilon_t &\thicksim  N(0,D_t.R_t.D_t) \notag \\
D_t^2 &\thicksim \text{Univariate GARCH(1,1) processes $\forall$ (i,j), i $\ne$ j} \notag \\
z_t&=D_t^{-1}.\varepsilon_t \notag \\
Q_t&={Q}(1-\alpha-\beta)+\alpha(z_t'z_t)+\beta(Q_{t-1}) \notag \\
R_t &= Diag(Q_t^{-1}).Q_t.Diag({Q_t}^{-1}) \notag \\
\end{align}

\hypertarget{results}{%
\section{Results}\label{results}}

This section presents the main empirical findings on measuring the
volatility spillover between US Dollar and EM and DM equity markets. It
uses the DCC-GARCH model results to highlight time-varying conditional
correlations between the dollar and these markets. This approach is
two-fold. Firstly the standard Univariate Garch (1,1) is modeled to give
the estimated volatility series (\(\sigma\) or \(h_t\) series\{ used
interchangeably in the analysis\}) whose its standardized residuals are
then utilized to estimate the Dynamic conditional correlations. The
series was first tested for conditional heteroskedasticity using the
Multivariate Generalization of the Lagrange Multiplier (LM) test (MARCH
test) of Engle (\protect\hyperlink{ref-engle1982}{1982}). The MARCH test
as detailed below shows that all the MV portmanteau tests rejected the
null hypothesis of no conditional heteroskedasticity as all the four
test statistics utilized have p-values of zero , affirming the choice to
use the chosen MVGARCH model. As expected the test confirmed the
presence of conditional heteroskedasticity in the weekly log return
series which is not surprising since financial time series data is known
to have conditional heteroskedasticity. This shows that the conditional
covariance of the sampled multivariate series is time dependent.

\begin{verbatim}
## Q(m) of squared series(LM test):  
## Test statistic:  1597.899  p-value:  0 
## Rank-based Test:  
## Test statistic:  736.862  p-value:  0 
## Q_k(m) of squared series:  
## Test statistic:  7212.863  p-value:  0 
## Robust Test(5%) :  2234.973  p-value:  0
\end{verbatim}

The resultant volatility estimates for each series are shown in figure
\ref{figure2}. The figure highlights high correlation between the
returns volatility of EM and DM equity markets which is highlighted
further during the GFC period as the volatility of all returns spiked.
This can be seen as a possible indication of a global systematic risk
factor. It is evident from the figure that EMs have the most volatile
returns compare to the DMs throughout the sample period. Among this,
MXEG is the most volatile, followed by MXBR indicating a significant and
distinct volatility effect within EMs that keeps growing overtime. The
BBDXY is the least volatile but among the return series the MXCA and
MXGB are among the least volatile. Overall the Univariate Garch (1,1)
model highlights the existence of a consistent and similar risk-return
relationship among EMs and DMs. The volatility in all returns appears to
have increased in the latter part of the sample period with more
dramatic swings beyond 2015. Generally, we find strong evidence of
time-varying volatility. Particularly, there is volatility clustering in
the sampled emerging markets as evidenced in most developed financial
markets amking the MVGARCH processes the most appropriated models to be
applied.

As mentioned above, the second step is then to use the standardized
residuals obtained from the estimated Univariate Garch Model to estimate
the time-varying DCC which provides a cleaner noise reduced version of
the dynamic co-movements between the US dollar and EM-DM returns. Figure
\ref{figure3} shows how the bivariate dynamic/time varying conditional
correlation between the USD and equity returns changes overtime..
Looking at the USD and MSCI index pairs in the figure we can see that
the USD average correlation with the equity market returns is relatively
low. Moreover, these volatility effects driven by the USD volatility can
be ranked to reflect the least and most sensitive markets to the USD
volatility. It is apparent from the figure that France followed by the
United Kingdom and Japan represent the most sensitive DM equity returns
to US dollar volatility. The volatility effect represents a
unidirectional spillover of the US dollar volatility to these markets
whose effect differs according to the individual market development and
degree of market integration. One explanation for the large effect in
these markets is the high use of the USD in commercial and financial
transactions thus increasing their foreign exchange risk premium. Most
of the emerging markets as portrayed in figure \ref{figure3} seem to be
less sensitive to the USD volatility where by India, Egypt and South
Africa are ranked lowest. This can be a result of the individual
characteristics of the EM countries. Interestingly there is still a
sharp decline in the correlations during the GFC indicating a
significant volatility spillover to all equity markets during the
period.

\hypertarget{conclusion}{%
\section{Conclusion}\label{conclusion}}

The paper examined the effect of U.S. dollar exchange rate on stock
market volatility. The sample period chosen contains periods of small
and large fluctuations such as the GFC which serves as an opportunity to
study the volatility effect on market behavior in both pre and post
crisis environments. The results highlight that heightened global
uncertainty during such periods amplifies the co-movement between equity
\& exchange rate markets' volatility across both developed and emerging
markets thus dampening investors' ability and desire to diversify their
portfolios. By relating the volatility effect in these markets we are
able to assess and differentiate whether the co-movement effects are a
result of a common factor or various macroeconomic factors beyond the
GFC having proven to be a common shock across the board. We can conclude
that there is significant evidence of low time varying correlations
between the USD and Equity returns. This implies that exchange rate and
equity markets are informationally inefficient such that one market has
some predictive power over the other. These results have substantial
implications for international portfolio managers, hedgers, individual
and institutional investors in the assessment of investment and asset
diversification decisions across markets. As a result Investors can
better comprehend how volatility between exchange rate and equity
markets interlink overtime in order to forecast market behavior.
Generally, the findings indicate a unidirectional spillover in dollar
volatility from the US to other markets whose impact differs according
to market development.

More importantly, the results show that investors may need to diversify
their investment portfolios and hedge against currency risk to maximize
returns and avoid the volatility risk that exchange rate markets impose
on equity markets. Specifically, international investors who do a large
portion of their business in U.S dollar traded assets would need to
hedge the USD-local currency exchange rate so that they lock in the
current exchange rate rather than be subjected to future volatility.
This will likely have a positive effect on the domestic asset returns
through decreased volatility and increased predictability. Some emerging
markets in particular still lack the requisite market development
equipped with hedging instruments that can mitigate potential exchange
rate risk exposure resulting in positive correlation between the
exchange rate volatility and equity markets volatility. However,
Emerging markets offer international investors with diversification
opportunities as evidenced by low correlations with USD volatility as
compared to developed markets. Investors with a long time horizon and
high tolerance for the volatility in emerging markets can expect returns
well in excess of developed market equities returns in the long run. In
particular, the findings show South Africa to have the lowest bivariate
conditional correlation with the USD amongst all the markets which can
present it as an investment haven for international portfolio managers
looking for assets with least co-movement characteristics.

As an avenue for future research, different measures of volatility such
as the VIX (measure of market variance/ fear index) can be explored to
see if the same bivariate conditional correlations exist between
Exchange rate market and equity markets in South Africa specifically as
has been evidenced with the GARCH derived USD volatility measure.

\newpage

\hypertarget{references}{%
\section*{References}\label{references}}
\addcontentsline{toc}{section}{References}

\hypertarget{refs}{}
\leavevmode\hypertarget{ref-adjasi2008}{}%
Adjasi, C., Harvey, S.K. \& Agyapong, D.A. 2008. Effect of exchange rate
volatility on the ghana stock exchange. \emph{African Journal of
Accounting, Economics, Finance and Banking Research}. 3(3).

\leavevmode\hypertarget{ref-bahmani1992}{}%
Bahmani-Oskooee, M. \& Sohrabian, A. 1992. Stock prices and the
effective exchange rate of the dollar. \emph{Applied economics}.
24(4):459--464.

\leavevmode\hypertarget{ref-bauwens2006}{}%
Bauwens, L., Laurent, S. \& Rombouts, J.V. 2006. Multivariate garch
models: A survey. \emph{Journal of applied econometrics}. 21(1):79--109.

\leavevmode\hypertarget{ref-bekaert1997}{}%
Bekaert, G. \& Harvey, C.R. 1997. Emerging equity market volatility.
\emph{Journal of Financial economics}. 43(1):29--77.

\leavevmode\hypertarget{ref-blitz2013}{}%
Blitz, D., Pang, J. \& Van Vliet, P. 2013. The volatility effect in
emerging markets. \emph{Emerging Markets Review}. 16:31--45.

\leavevmode\hypertarget{ref-bollerslev1994}{}%
Bollerslev, T., Engle, R.F. \& Nelson, D.B. 1994. ARCH models.
\emph{Handbook of econometrics}. 4:2959--3038.

\leavevmode\hypertarget{ref-chkili2014}{}%
Chkili, W. \& Nguyen, D.K. 2014. Exchange rate movements and stock
market returns in a regime-switching environment: Evidence for brics
countries. \emph{Research in International Business and Finance}.
31:46--56.

\leavevmode\hypertarget{ref-desantis1994}{}%
DeSantis, G. \& Imrohoroglu, S. 1994. Stock returns and volatility in
emerging markets. \emph{University of Southern California working
paper}.

\leavevmode\hypertarget{ref-engle2002}{}%
Engle, R. 2002. Dynamic conditional correlation: A simple class of
multivariate generalized autoregressive conditional heteroskedasticity
models. \emph{Journal of Business \& Economic Statistics}.
20(3):339--350.

\leavevmode\hypertarget{ref-engle1982}{}%
Engle, R.F. 1982. Autoregressive conditional heteroscedasticity with
estimates of the variance of united kingdom inflation.
\emph{Econometrica: Journal of the Econometric Society}. 987--1007.

\leavevmode\hypertarget{ref-hung2019}{}%
Hung, N.T. 2019. Return and volatility spillover across equity markets
between china and southeast asian countries. \emph{Journal of Economics,
Finance and Administrative Science}.

\leavevmode\hypertarget{ref-kanas2000}{}%
Kanas, A. 2000. Volatility spillovers between stock returns and exchange
rate changes: International evidence. \emph{Journal of Business Finance
\& Accounting}. 27(3-4):447--467.

\leavevmode\hypertarget{ref-katzke2013}{}%
Katzke, N. 2013. South african sector return correlations: Using dcc and
adcc multivariate garch techniques to uncover the underlying dynamics.
\emph{South African Sector Return Correlations: using DCC and ADCC
Multivariate GARCH techniques to uncover the underlying dynamics}.

\leavevmode\hypertarget{ref-katzke2020}{}%
Katzke, N.F. 2020. {[}Online{]}, Available:
\url{https://www.fmx.nfkatzke.com/posts/2020-08-15-theory5/Notes/Session_5.pdf/}.

\leavevmode\hypertarget{ref-kumar2013}{}%
Kumar, M. 2013. Returns and volatility spillover between stock prices
and exchange rates. \emph{International Journal of Emerging Markets}.

\leavevmode\hypertarget{ref-lim2014}{}%
Lim, S.Y. \& Sek, S.K. 2014. Exploring the inter-relationship between
the volatilities of exchange rate and stock return. \emph{Procedia
Economics and Finance}. 14:367--376.

\leavevmode\hypertarget{ref-mcgibany1995}{}%
Mcgibany, J.M. \& Nourzad, F. 1995. Exchange rate volatility and the
demand for money in the us. \emph{International Review of Economics \&
Finance}. 4(4):411--425.

\leavevmode\hypertarget{ref-mishra2007}{}%
Mishra, A.K., Swain, N. \& Malhotra, D.K. 2007. Volatility spillover
between stock and foreign exchange markets: Indian evidence.
\emph{International journal of business}. 12(3).

\leavevmode\hypertarget{ref-morales2008}{}%
Morales, L. 2008. Volatility spillovers between stock returns and
foreign exchange rates: Evidence from four eastern european countries.

\leavevmode\hypertarget{ref-raghavan2008}{}%
Raghavan, M. \& Dark, J. 2008. Return and volatility spillovers between
the foreign exchange market and the australian all ordinaries index.
\emph{The IUP Journal of Applied Finance}. 14(1):41--48.

\leavevmode\hypertarget{ref-ruppert2015}{}%
Ruppert, D. \& Matteson, D.S. 2015. \emph{Statistics and data analysis
for financial engineering}. ed. Springer.

\leavevmode\hypertarget{ref-shephard1996}{}%
Shephard, N. 1996. Statistical aspects of arch and stochastic
volatility. \emph{Monographs on Statistics and Applied Probability}.
65:1--68.

\leavevmode\hypertarget{ref-shin2005}{}%
Shin, J. 2005. Stock returns and volatility in emerging stock markets.
\emph{International Journal of Business and economics}. 4(1):31.

\leavevmode\hypertarget{ref-silvennoinen2009}{}%
Silvennoinen, A. \& Teräsvirta, T. 2009. Multivariate garch models. In
ed. Springer \emph{Handbook of financial time series}. 201--229.

\leavevmode\hypertarget{ref-tsay2013}{}%
Tsay, R.S. 2013. \emph{Multivariate time series analysis: With r and
financial applications}. ed. John Wiley \& Sons.

\newpage

\hypertarget{appendix-a}{%
\section*{Appendix A}\label{appendix-a}}
\addcontentsline{toc}{section}{Appendix A}

\hypertarget{returns-description-table}{%
\subsection{Returns Description Table}\label{returns-description-table}}

\begin{table}[H]
\centering
\begin{tabular}{rll}
  \hline
 & Ticker & Description \\ 
  \hline
1 & MXBR & Brazil \\ 
  2 & MXCA & Canada \\ 
  3 & MXCH & Switzerland \\ 
  4 & MXFR & France \\ 
  5 & MXGB & United Kingdom \\ 
  6 & MXIN & India \\ 
  7 & MXJP & Japan \\ 
  8 & MXZA & South Africa \\ 
  9 & TAMSCI & Taiwan \\ 
  10 & MXEG & Egypt \\ 
  11 & BBDXY & Bloomberg US Dollar Spot Index \\ 
   \hline
\end{tabular}
\caption{Returns description Table \label{tab1}} 
\end{table}

\hypertarget{msci-daily-returns-series}{%
\subsection{MSCI Daily Returns Series}\label{msci-daily-returns-series}}

\begin{figure}[H]

{\centering \includegraphics{Impact-of-Dollar-Volatility-on-EM-DM-Equity-Markets_files/figure-latex/figure1-1} 

}

\caption{MSCI Returns \label{figure1}}\label{fig:figure1}
\end{figure}
\newpage

\hypertarget{apendix-b}{%
\section*{Apendix B}\label{apendix-b}}
\addcontentsline{toc}{section}{Apendix B}

\hypertarget{individual-standard-univariate-garch11-model}{%
\subsection{Individual Standard Univariate GARCH(1,1)
model}\label{individual-standard-univariate-garch11-model}}

\begin{figure}[H]

\includegraphics{Impact-of-Dollar-Volatility-on-EM-DM-Equity-Markets_files/figure-latex/figure2-1} \hfill{}

\caption{Series Volatility \label{figure2}}\label{fig:figure2}
\end{figure}

\hypertarget{dynamic-conditional-correlations}{%
\subsection{Dynamic Conditional
Correlations}\label{dynamic-conditional-correlations}}

\begin{figure}[H]

\includegraphics{Impact-of-Dollar-Volatility-on-EM-DM-Equity-Markets_files/figure-latex/figure3-1} \hfill{}

\caption{Series Volatility \label{figure3}}\label{fig:figure3}
\end{figure}

\bibliography{Tex/ref}





\end{document}
